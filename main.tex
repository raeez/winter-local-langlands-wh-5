\documentclass[12pt]{amsart}

%%%%%%%%%%%%%%%%%%%%%%%%
%% project modularity %%
%%
\usepackage[subpreambles=true]{standalone}
\usepackage{import} % more elegant than \input for standalone documents

%%%%%%%%%%%%%%%%%%%%%%%%%%%%%%%%%%
%% design, formatting and fonts %%
%%

%\usepackage[titletoc]{appendix}
%\usepackage[toc,page]{appendix}
\usepackage{amsmath,amssymb,amsthm,amsrefs}
\usepackage{mathtools} % TODO find better for \coloneqq (':=')
\usepackage{mathpazo}
\usepackage{inconsolata}
%\usepackage{euler}
\usepackage{epigraph} % TODO how does this work
\usepackage{showkeys} % TODO http://texdoc.net/texmf-dist/doc/latex/tools/showkeys.pdf
\usepackage{etoolbox}
\usepackage{ifthen} % one route to argument overloading

%%%%%%%%%%%%%%%%%%
%% Nomenclature %%
%%
%\usepackage[intoc]{nomencl}
\usepackage{nomencl}
\makenomenclature

%%%%%%%%%%%%%%
%% graphics %%
%%
%% TODO https://www.sharelatex.com/learn/Inserting_Images
\usepackage{graphicx}
\usepackage{float}
\graphicspath{{graphics/}{../graphics/}} % relative to both / and sections/

%%%%%%%%%%%%%%%%%%%%%%%%%%%%%%%%%%%
%% TODO learn correct usage of tikz
%% tikz
\usepackage{tikz-cd}
%% Functions
%% \begin{tikzcd}[column sep= small,row sep=0ex]
%%     M_f \colon \pi_1(S \smallsetminus \{y_1, \dots, y_n\}, y) \arrow[r]& \Bij(f^{-1}(y)) \\
%%    \gamma \arrow[r, mapsto]                                   & M_f(\gamma) = \sigma_{\gamma}^{-1}
%% \end{tikzcd}
%%
%% \begin{align*}
%%   M_f \colon \pi_1(S \smallsetminus \{y_1, \dots, y_n\}, y) & \longrightarrow \Bij(f^{-1}(y)) \\
%%   \gamma & \longmapsto M_f(\gamma) = \sigma_\gamma^{-1}
%% \end{align*}
%%
%% Or this:\medskip
%%
%% Let $ S' =S \smallsetminus \{y_1, \dots, y_n\} $. Define
%% $ \begin{aligned}[t]
%% M_f \colon \pi_1(S', y) &\longrightarrow \Bij(f^{-1}(y)) \\
%% \gamma &\longmapsto M_f(\gamma) = \sigma_\gamma^{-1}
%% \end{aligned} $


%%%%%%%%%%%%%%%%%%%%%%%%%%%%%%%%%%%%%%%%%%%%%%%%%%
%% TODO find a better way to handle subsections %%
%% - glg.tex file inclusions
%% - introduce nomenclature db using section tags + groups
%% - standardize labels throughout project
%% - place the above into an array and autogen glg.tex
%% - glg.tex file inclusions
%% - introduce nomenclature db using section tags + groups
%% - standardize labels throughout project

% \usepackage{etoolbox}
%  \renewcommand\nomgroup[1]{%
%    \item[\bfseries
%      \ifstrequal{#1}{\leca}{\lecatitle}{%
%      \ifstrequal{#1}{\lecb}{\lecbtitle}{%
%      \ifstrequal{#1}{\lecc}{\lecctitle}{%
%      }}}%
%   ]}
%%%%%%%%%%%%%%%%%%%%%%%
%% Table of contents %%
%%
%% TODO
%% - fix nomenclature
%%
%% \makeatletter
%% \def\thenomenclature{%
%%   \section*{\nomname}
%%   \if@intoc\addcontentsline{toc}{section}{\nomname}\fi%
%% \nompreamble
%% \list{}{%
%% \labelwidth\nom@tempdim
%% \leftmargin\labelwidth
%% \advance\leftmargin\labelsep
%% \itemsep\nomitemsep
%% \let\makelabel\nomlabel}}
%% \makeatother


%%%%%%%%%%%%%%%%%%%%%%%%%%
%% Formatting / Display %%
%%
% \newcommand{\HRule}{\rule{\linewidth}{0.5mm}}
\numberwithin{equation}{section}
% TODO understand what this does
% (something like number equations within sections)

%%%%%%%%%%%%%
%% urls/links
% \usepackage{hyperref}

% examples
% c.f. \hyperref[mainlemma1]{lemma \ref*{mainlemma} }.
% take a look at my website \url{http://raeez.com}
% it never hurts to \href{http://wiki.org/RTFM}{RTFM}
% I can be reached at
% \href{mailto:this_is_a_false_addr@raeez.com}{this\_is\_a\_false\_addr@raeez.com}

%%%%%%%%%%%%%%%%%%%%%%%%%%
%% Theorem Environments %%
%%
\newtheorem{thm}{Theorem}[section]
\newtheorem{prop}[thm]{Proposition}
\newtheorem{lem}[thm]{Lemma}
\newtheorem{cor}[thm]{Corollary}
\theoremstyle{remark}
\newtheorem{rmk}[thm]{Remark}
\theoremstyle{definition}
\newtheorem{defn}[thm]{Definition}
\newtheorem{ex}[thm]{Example}
\newtheorem{exc}[thm]{Exercise}
\newtheorem{conj}[thm]{Conjecture}
\newtheorem{prob}[thm]{Problem}
\newtheorem{oprob}[thm]{Open Problem}
\newtheorem{stmt}[thm]{Statement}
\newtheorem{qn}[thm]{Question}
\newtheorem{ans}[thm]{Answer}

%%%%%%%%%%%%%%%%%%
%% Nomenclature %%
%%
%% TODO figure out better solution
%% https://tex.stackexchange.com/questions/361373/nomenclature-entry-in-toc-not-indented-like-a-chapter/361376

%%%%%%%%%%%%%%%%%%%%%%%%%%%%%%%%%%%%
%% Modify nomenclature generation %%
%% 1. SI Units
%% 2. titled groups
%% 3. enforce manual order

%% 1. Enable SI units
%% \usepackage{siunitx}
%% \sisetup{
%% inter-unit-product=\ensuremath{{}\cdot{}},
%% per-mode=symbol
%% }
%% \nc{\nomunit}[1]{\renewcommand{\nomentryend}{\hspace*{\fill}#1}}

%% c.f. https://tex.stackexchange.com/questions/118114/commands-that-may-take-a-variable-number-of-arguments


%% 2. titled groups

%% TODO 3. manual ordering

%% TODO why is % often used before a newline?

%% TIP wrap long descriptions in a \parbox e.g.
%% \nm[x]{$x$}{\parbox[t]{.75\textwidth}{Unknown variable with a very very
%% very very very very very very long description}\nomunit{\si{\second}}}

%% The following implements grouping in the nomenclature preamble
%% c.f. 1. https://tex.stackexchange.com/questions/166556/how-to-make-a-clean-and-grouped-nomenclature-list
%%      2. https://tex.stackexchange.com/questions/310128/grouped-nomenclature
%%      3. https://tex.stackexchange.com/questions/318850/grouping-nomenclature-elements
%%      4. https://www.sharelatex.com/learn/Nomenclatures

%% \begin{document}
%% \mbox{}
%%
%% \nm[A, 02]{$c$}{Speed of light in a vacuum inertial system}
%% \nm[A, 03]{$h$}{Plank Constant}
%% \nm[A, 01]{$g$}{Gravitational Constant}
%% \nm[B, 03]{$\mathbb{R}$}{Real Numbers}
%% \nm[B, 02]{$\mathbb{C}$}{Complex Numbers}
%% \nm[B, 01]{$\mathbb{H}$}{Octonions}
%% \nm[C]{$V$}{Constant Volume}
%% \nm[C]{$\rho$}{Friction Index}

%\input{mathmacros.tex}
%\author{Raeez Lorgat}
\email{raeez@mit.edu}
\urladdr{http://math.raeez.com}


\newcommand\Wh[1]{\textsc{Wh}-#1}
\newcommand\Ja[1]{\textsc{Ja}-#1}
\title{Winter Local Langlands \textsc{\Wh{5}}}


\newcommand\LT{\mathfrak{LT}}
\newcommand\LN{\mathfrak{LN}}
\newcommand\LNm{\mathfrak{L}(\mathfrak{N}^-)}
\newcommand\Lnm{\mathfrak{L}(\mathfrak{n}^-)}
\newcommand\LG{\mathfrak{LG}}
\newcommand\Bun{\mathbf{Bun}}
\newcommand\afftmod{\hat{\mathfrak{t}}-\mathbf{Mod}}
\newcommand\affgmod{\hat{\mathfrak{g}}-\mathbf{Mod}}
\newcommand\Gr{\mathbf{Gr}}
\newcommand\KL{\mathbf{KL}}
\newcommand\Whit{\mathbf{Whit}}
\newcommand\neglv{-\kappa}
\newcommand\poslv{\kappa}
\newcommand\dneglv{-\check{\kappa}}
\newcommand\dposlv{\check{\kappa}}

\begin{document}

  %%%%%%%%%%%%%%%%
  %% Title etc. %%
  %%

\begin{abstract}The purpose of this talk (\Wh{5}) is two-fold: first we will set
up the context needed to attack the FLE; second, we will address material needed
for the local-to-global compatibilities to be discussed in \Ja{4}.\end{abstract}

  \maketitle
  %\tableofcontents
  %\mbox{}
  %\nomenclature{$\Klocfrac$}{The field
of laurent series valued in $K$; equivalently, the fraction field of
series valued in $k$; equivalently, the ring of integers of the
completed local field.}
\nomenclature{$\Z$}{The ring of integers.}
\nomenclature{$\Q$}{The rational number field}
\nomenclature{$\R$}{The real number field}
\nomenclature{$\C$}{The complex number field}
\nomenclature{$\Spec{A}, \Specf{A\fpowerser}$}{The respective Prime and formal spectra of the commutative rings $A$ and $A\fpowerser$}
\nomenclature{$I,\idshf{I}$}{For an ideal $I$ in a commutative ring $A$,
$\idshf{I}$ denotes the associated ideal sheaf on $\Spec{A}$}
\nomenclature{$\projline$}{The projective line as algebro-geometric object, for
example, as represented in the functor-of-points yoga by two copies of $\Z[x]$
glued along a common $\Z\laurentpoly$ }
\nomenclature{$\K$}{The abstract total field}
\nomenclature{$\polyring{R}{n}$}{The ring of $R$-valued polynomials in the
formal variables $z_1,\ldots,z_n$ valued in the ring $R$.}
% TODO clarify distinction between polynomials 'in' and 'over' the $z_i$.
% what is the same: choose a convention for the symmetric algebra and its dual
\nomenclature{$ \FnS $}{Function field of a curve $\Sigma$ defined over $\grk$ }
\nomenclature{$\classgrp{\grk}$}{The class formation group associated to $\grk$.}
\nomenclature{$\F_p$}{A finite field of $p$ elements; for example the
quotient ring $\Z \ov{\idealgenby{p}}$}
\nomenclature{$\D = \D_a$}{The formal disc defined over $\grk$ with the
distinguished $\pt = a$ labeling the origin.}
\nomenclature{$\Gr$ }{The affine grassmannian of $G$.}
\nomenclature{$\Ga$}{The additive group as commutative reductive algebraic group}
\nomenclature{$\Gm$}{The multiplicative group as commutative reductive algebraic group.}
\nomenclature{$\lpgrass$}{The classical loop grassmannian of $G$.}
\nomenclature{$\glpgrass$}{The \textit{generalized} loop grassmannian of $G$}
\nomenclature{$\Flds$}{The Category of Fields}
\nomenclature{$\cartdual$}{Cartier duality on a category } % TODO say more}
\nomenclature{$\Mat{n}\grk$}{$n\times n$ matrices with coefficients in $\grk$}
\nomenclature{$\RH^*$}{Right derived functor of homology}
\nomenclature{$X\tateshiftedby{n}$}{The nth Tate Shift}
\nomenclature{$\TateCat$}{The tate category associated to a group $H$, defined as the
categorical quotient $\mathcal{A}^G_{N_G(\mc{A})}$} %N_G(\mathcal{A}
\nomenclature{$\B{G}$}{everyone's favourite quotient category}
\nomenclature{$\ptmod{G}$}{everyone's favourite quotient category}
%\nomenclature{$\hcoH{G,A,n}$}{blah}
\nomenclature{$\Vec_{\grk}$}{Category of Vector Spaces over $\grk$}
\nomenclature{$\divisor{p},\divisoridshf{\divisor{p}}$}{A divisor on some ambient space
along with its ideal sheaf}
\nomenclature{$\AJ$}{The Abel Jacobi map taking a divisor $\divisor{p}$
  supported on a space $X$ to $\AJ: \divisor{p} \mapsto \ringint_{X}
(-\divisor{p})\divisoridshf{I_{\divisor{p}}}$}
\nomenclature{$\IndProSch$}{The category of $\Ind$-$\Pro$-Schemes}
\nomenclature{$\FinIndProSch$}{The category of \textit{finite} $\IndPro$-Schemes}
\nomenclature{$\FinIndSch$}{The category of \textit{finite} $\Ind$-Schemes}
\nomenclature{$\A$}{The category of commutative indschemes over a ring $\grk$}
\nomenclature{$\Zp$}{The inverse limit of the inverse system of rings $\Z \ov{p^n \Z}$,
also called the formal completion at the point $p$ in the curve $\Spec{\Z}$}
\nomenclature{$\Set$}{The (small) category of Sets}
\nomenclature{$\Set$}{The (small) category of Sets}
\nomenclature{$\Hilb_{X}$}{The hilbert scheme of $X$ representing the moduli of finite length subschemes of $X$.}
%\nomenclature{$\shfcoH{X,\shf{I},n}$}{The $n$th cohomology of the sheaf $\shf{I}$ over the space $X$}.
\nomenclature{$\grpring{Z}{G}$}{The group ring of $G$}
\nomenclature{$\Bun G$}{The moduli space of $G$ bundles.}
\nomenclature[coh]{$\dualmod{M}$}{The dual module $\Hom{A}(M,A)$ associated to an object $M \in \ModCat{A}$ for some ring $A$}
\nomenclature{$\LModCat{A}$}{The category of left modules over a ring $A$.}
\nomenclature{$\RModCat{A}$}{The category of right modules over a ring $A$.}
\nomenclature{$\Schemes$}{The category of Schemes}
\nomenclature{$\NoethSch$}{The category of Noetherian Schemes}
\nomenclature{$\NoethIntSch$}{The category of Integral Noetherian Schemes}
\nomenclature{$\Sch{\grk}$}{The category of Schemes defined over $\grk$}
\nomenclature{$\IndSch{\grk}$}{The category of IndSchemes defined over $\grk$}
\nomenclature{$\ModCat{A}$}{The category of modules over a ring $A$.}
\nomenclature{$\SL_2$}{The reductive affine algebraic group scheme
  associated to the Special Linear Group of invertible determinant $1$ matrices
  i.e.\ the fiber $det^{-1}(-1)$ in $\Mat{2}$. Equivalently characterized via a
  \textit{functor of points} formalism e.g.\ $\grkAlg \rightarrow \Set$
represented in $\ZAlg$ by $\grpring{\Z}{\SL_2} \isom$ }
  %\polyring{Z}{4}\ov{\idealgenby{z_1z_3 - z_2z_4 - 1}}$}
\nomenclature{$G$}{A reductive group.}
\nomenclature{$\ringint$}{The local ring $\klocint$ ring of formal power series.}
% TODO frame as localization / indsystem
\nomenclature{$\grk$}{The ground field,
  often the complex numbers $\C$, and
  almost always embedded in the total field
  $K$.}
%\nomenclature[\ageom]{$\Spec{A}, \Specf{A\fpowerser}$}{The respective Prime and formal spectra of the commutative rings $A$ and $A\fpowerser$}
%\nomenclature{$I,\idshf{I}$}{For an ideal $I$ in a commutative ring $A$,
%$\idshf{I}$ denotes the associated ideal sheaf on $\Spec{A}$}
%\nomenclature{$\projline$}{The projective line as algebro-geometric object, for
%example, as represented in the functor-of-points yoga by two copies of $\Z[x]$
%glued along a common $\Z\laurentpoly$ }
%\nomenclature[\seccfm]{$\K$}{The abstract total field}
%\nomenclature[\ageom]{$\polyring{R}{n}$}{The ring of $R$-valued polynomials in the
%formal variables $z_1,\ldots,z_n$ valued in the ring $R$.}
%\nomenclature{$G$}{A reductive group.}
% algebraic fields, formal power series and their fraction fields
%\nomenclature[\ageom]{$\ringint$}{The local ring $\klocint$ ring of formal power
%series valued in $k$; equivalently, the ring of integers of the
%completed local field.}
%\nomenclature[\ageom]{$\grk$}{The ground field,
%  often the complex numbers $\C$, and
%  almost always embedded in the total field
%  $K$.}
%\nomenclature[\ageom]{$\Klocfrac$}{The field
%  of laurent series valued in $K$; equivalently, the fraction field of
%$\ringint$.}
%\nomenclature{$\Z$}{The ring of integers}
%\nomenclature{$\Q$}{The rational number field}
%\nomenclature{$\R$}{The real number field}
%\nomenclature{$\C$}{The complex number field}
%\nomenclature{$ \FnS $}{Function field of a curve $\Sigma$ defined over $\grk$ }
%\nomenclature{$\classgrp{\grk}$}{The class formation group associated to $\grk$.}
%\nomenclature{$\F_p$}{A finite field of $p$ elements; for example the
%\nomenclature{$\D = \D_a$}{The formal disc defined over $\grk$ with the
%distinguished $\pt = a$ labeling the origin.}
%\nomenclature{$\Gr$ }{The affine grassmannian of $G$.}
%\nomenclature{$\Ga$}{The additive group as commutative reductive algebraic group}
%\nomenclature{$\Set$}{The (small) category of Sets}
%\nomenclature{$\Qp$}{The fraction field of the ring of $p$-adic integers, i.e. $\Frac{\Zp}$}
%\nomenclature{$\Zp$}{The inverse limit of the inverse system of rings $\Z \ov{p^n \Z}$,
%\nomenclature[\seccfm]{$\A$}{The category of commutative indschemes over a ring $\grk$}
%\nomenclature[\seccfm]{$\FinIndSch$}{The category of \textit{finite} $\Ind$-Schemes}
%\nomenclature[\seccfm]{$\FinIndProSch$}{The category of \textit{finite} $\IndPro$-Schemes}
%\nomenclature{$\Hilb_{X}$}{The hilbert scheme of $X$ representing the moduli of finite length subschemes of $X$.}
%%\nomenclature{$\shfcoH{X,\shf{I},n}$}{The $n$th cohomology of the sheaf $\shf{I}$ over the space $X$}.
%\nomenclature[\seccfm]{$\Vec_{\grk}$}{Category of Vector Spaces over $\grk$}
%\nomenclature[\seccfm]{$\divisor{p},\divisoridshf{\divisor{p}}$}{A divisor on some ambient space
%along with its ideal sheaf}
%\nomenclature[\seccfm]{$\AJ$}{The Abel Jacobi map taking a divisor $\divisor{p}$
%  supported on a space $X$ to $\AJ: \divisor{p} \mapsto \ringint_{X}
%(-\divisor{p})\divisoridshf{I_{\divisor{p}}}$}
%\nomenclature[\seccfm]{$\IndProSch$}{The category of $\Ind$-$\Pro$-Schemes}
%\nomenclature[\seccoh]{$\grpring{Z}{G}$}{The group ring of $G$}
%\nomenclature[\seccoh]{$\Bun G$}{The moduli space of $G$ bundles.}
%\nomenclature[coh]{$\dualmod{M}$}{The dual module $\Hom{A}(M,A)$ associated to an object $M \in \ModCat{A}$ for some ring $A$}
%\nomenclature{$\LModCat{A}$}{The category of left modules over a ring $A$.}
%\nomenclature{$\RModCat{A}$}{The category of right modules over a ring $A$.}
%\nomenclature[\secindobj]{$\Schemes$}{The category of Schemes}
%\nomenclature[\secgaugetheory]{$\SL_2$}{The reductive affine algebraic group scheme
%  associated to the Special Linear Group of invertible determinant $1$ matrices
%  i.e.\ the fiber $det^{-1}(-1)$ in $\Mat{2}$. Equivalently characterized via a
%  \textit{functor of points} formalism e.g.\ $\grkAlg \rightarrow \Set$
%  represented in $\ZAlg$ by $\grpring{\Z}{\SL_2} \isom
%  \polyring{Z}{4}\ov{\idealgenby{z_1z_3 - z_2z_4 - 1}}$}
%\nomenclature[\secindobj]{$\NoethSch$}{The category of Noetherian Schemes}
%\nomenclature[\secindobj]{$\NoethIntSch$}{The category of Integral Noetherian Schemes}
%\nomenclature[\secindobj]{$\Sch{\grk}$}{The category of Schemes defined over $\grk$}
%\nomenclature[\secindobj]{$\IndSch{\grk}$}{The category of IndSchemes defined over $\grk$}
%\nomenclature[\secindobj]{$\ModCat{A}$}{The abelian category of modules over a
%%%%%%%%%%%%%%%%%%%%%%%%%%
%% %\nomenclature[\secgaugetheory] %%
%%
%%
%%\nomenclature{$\hcoH{G,A,n}$}{blah}
%\nomenclature{$\Gm$}{The multiplicative group as commutative reductive algebraic group.}
%\nomenclature{$\lpgrass$}{The classical loop grassmannian of $G$.}
%\nomenclature{$\glpgrass$}{The \textit{generalized} loop grassmannian of $G$}
%\nomenclature[\seccfm]{$\Flds$}{The Category of Fields}
%\nomenclature[\seccfm]{$\cartdual$}{Cartier duality on a category } % TODO say more}
%\nomenclature{$\Mat{n}\grk$}{$n\times n$ matrices with coefficients in $\grk$}
%\nomenclature{$\RH^*$}{Right derived functor of homology}
%\nomenclature{$X\tateshiftedby{n}$}{The $n$th Tate Shift}
%\nomenclature{$\TateCat$}{The tate category associated to a finite
%  abelian group $H$, defined as the categorical quotient
%  $\mathcal{A}^G_{N_G(\mathcal{A}}$}
%\nomenclature{$\B{G}$}{everyone's favourite quotient category}
%\nomenclature{$\ptmod{G}$}{everyone's favourite quotient category}
\nomenclature{$\DMod (X)$}{The category $\DMod$ of $D$-modules defined on $X$}
\nomenclature{$\IndSt$}{The Category of $\Ind$-Stacks}
\nomenclature{$\IndNSt$}{The Category of $\Ind$-$N$-Stacks}

  %\printnomenclature

  %%%%%%%%%%%%%%%%%%%%%%
  %% body of document %%
  %%


  % \begin{rmk}
  %   This talk has been cut down due to a change in the planned exposition.
  % \end{rmk}

\section{The formal setting}

  We begin by addressing the formal setting: Recall that we have introduced two
  forms of the FLE (at positive and negative levels
  respectively).\\


  We begin at the negative level\footnotemark; when we set out to prove the FLE
  we want to relate the Kazhdan-Lusztig category for $G$ and Whittaker category
  for $\Gr_{\check{G}}$ to the corresponding categories for the Torus; this
  relation takes the form of a diagram:

  \begin{tikzcd}
    \KL(G)_{\neglv} \arrow[d,"j_!^{KM,Lus}"] \arrow[r,"FLE_G" description]
     & \Whit_{\dneglv}(\Gr_{\check{G}}) \arrow[d, "j_!^{Whit,Lus}"] \\
     \KL(T)_{\neglv} \arrow[r, "\sim", "FLE_T"] & \Whit_{\dneglv}(\Gr_{\check{T}}) \\
  \end{tikzcd}

  \footnotetext{Recall that by convention
    $\kappa$ denotes positive level, hence $-\kappa$ is used for negative
  level.}

  The desired vertical functors should make the diagram commute, be factorizable
  and by applying these functors to the units on both sides we expect to obtain
  the factorization algebras constituting the subject of the last talk.
  Morevover, the factorizable algebras should match up under the bottom
  isomorphism, but it will \textbf{not} be the case that the induced map on
  factorization module categories will be an equivalence.
  % insert some discussion

  The essential thrust of the last talk will be a conjectural bootstrapping of
  an equivalence given what structure we do have apparent here.

  \begin{rmk}
    We saw in \Ja{3} the definition of the functor $j_!^{KM,Lus}$. Note the
    designation of \textit{Lus} in contrast to the existence of a more naive
    functor corresponding to the de-Concini-Kac form of the quantum group, which
    will not produce the correct diagram. %more commentary
  \end{rmk}

  % TODO check to see if Lin's notes defines both Whit and KM sides

  We still need a definition of the functor $j_!^{Whit,Lus}$. First, we'll drop
  the notation \textit{Lus}\footnotemark\footnotetext{reserving this notation
    for the factorization algebra as in $\Omega_q^{Lus}$ arising as the image of
  the unit under the functor $j_!^{KM}$} and define another functor
  $j_{!*}^{Whit}$; this new functor also
  has an incarnation on the Kac-Moody side as $j_{!*}^{KM}$, giving a diagram:

  \begin{tikzcd}
    \KL(G)_{\neglv} \arrow[transform
    canvas={xshift=0.9ex},d,"j_{!*}^{KM}"]\arrow[transform
    canvas={xshift=-0.4ex}, d,"j_!^{KM}"] \arrow[r,"FLE_G" description]
    & \Whit_{\dneglv}(\Gr_{\check{G}}) \arrow[transform
    canvas={xshift=0.9ex},d, "j_{!*}^{Whit}"]\arrow[transform
    canvas={xshift=-0.4ex}, d, "j_!^{Whit}"] \\
     \KL(T)_{\neglv} \arrow[r, "\sim", "FLE_T"] & \Whit_{\dneglv}(\Gr_{\check{T}}) \\
  \end{tikzcd}

  We can similarly express the corresponding diagram at positive level:

  \begin{tikzcd}
    \KL(G)_{\poslv} \arrow[transform
    canvas={xshift=0.9ex},d,"j_{!*}^{KM}"]\arrow[transform
    canvas={xshift=-0.4ex}, d,"j_!^{KM}"] \arrow[r,"\tilde{FLE}_G" description]
    & \Whit_{\dposlv}(\Gr_{\check{G}}) \arrow[transform
    canvas={xshift=0.9ex},d, "j_{!*}^{Whit}"]\arrow[transform
    canvas={xshift=-0.4ex}, d, "j_!^{Whit}"] \\
    \KL(T)_{\poslv} \arrow[r, "\sim", "\tilde{FLE}_T"] & \Whit_{\dposlv}(\Gr_{\check{T}}) \\
  \end{tikzcd}

  As we have seen, for the torus there is no disctinction between FLE for
  positive and negative levels (recall positivity refers to the killing form on
  simple factors.) However, on the Kac-Moody side, these functors will differ by
  a Cartan involution. In contrast, since the Whittaker category is geometric,
  it does not see the distinction between positive and negative level; the main
  features of these functors on the Whittaker side will be the same as at the
  negative level. As we saw in \Ja{3}, these functors on the Kac-Moody
  side strongly depend on the positivity or negativity of the level, while on
  the Whittaker side they do not.\\

  \begin{rmk}
    What was referred to as $j_*$ in \Ja{3} (in relation to $C_*$ semi-infinite
    cohomology) at positive level is here $j_!$.
    % TODO clarify
  \end{rmk}

  In the next section we will recall definitions of a subset of these functors
  (for the rest, and in greater detail, see \Ja{3}).

  \begin{rmk}
    We can express the plan for \Ja{4} in terms of these FLE diagrams for
    negative and positive level: to see this, note that all of these local
    categories are related to the corresponding global categories
    ($\Bun(G),\Bun(\check{G}) \Bun(T),\Bun(\check{T})$, and similarly for
    parabolics and their levi factors etc.\ ), such that there analogously
    exist functors relating these global categories. \Ja{4} will describe the
    interaction of these global categories equipped with global functors and
    their compatabilities and interactions with variants of the Eisenstein
    series and constant term functors. We will see that the structure amounts
    in the global setting to a series of commutative cubes.
  \end{rmk}

\section{Recollections on definitions of the functors $j_!$ and $j_{!*}$}
\subsection{$j_{\{!,!*\}}^{KM}$ at the positive level}
Let us begin by recalling from \Ja{3} the
semi-infinite cohomology functor

  $$C_* = C^{\frac{\infty}{2}}(\Lnm^+, -),$$

  in terms of which we can define $j_!^{KM}$ and $j_{!*}^{KM}$, via the following diagram:

% TODO swap to more traditional N^-(K)T(O) notation?
  \begin{tikzcd}
    j_{\{!,!*\}}^{KM} : \affgmod_{\poslv}^{\LG^+} \ar[equal, d] \arrow[r, transform
    canvas={yshift=0.9ex}, "\phi_!"] \arrow[r, transform
    canvas={yshift=-0.4ex}, "\phi_{!*}"]&
    (\affgmod_{\poslv})_{\LNm\LT^+} \arrow[r, "C_*"] &
      (\afftmod_{\poslv})_{\LT^+} \ar[equal, d] \\
    \KL(G)_{\poslv}  & & \KL(T)_{\poslv} \\
  \end{tikzcd}

Here we have used the fact that invariants and co-invariants are the same, as
well as the fact that $C_*$ is well-behaved at the positive level. Thus we have
reduced the definition of $j_!^{KM}$ and $j_{!*}^{KM}$ to the specification of
$\phi_!$ and $\phi_{!*}$.

\end{document}
