\documentclass[12pt]{amsart}

\usepackage{amsmath,amssymb,amsthm,amsrefs}
\usepackage{tikz-cd}
\newtheorem{thm}{Theorem}[section]
\newtheorem{rmk}[thm]{Remark}

\newcommand\Wh[1]{\textsc{Wh}-#1}
\newcommand\Ja[1]{\textsc{Ja}-#1}
\newcommand\LT{\mathfrak{LT}}
\newcommand\LN{\mathfrak{LN}}
\newcommand\LNm{\mathfrak{L}(\mathfrak{N}^-)}
\newcommand\Lnm{\mathfrak{L}(\mathfrak{n}^-)}
\newcommand\LG{\mathfrak{LG}}
\newcommand\Bun{\mathbf{Bun}}
\newcommand\afftmod{\hat{\mathfrak{t}}-\mathbf{Mod}}
\newcommand\affgmod{\hat{\mathfrak{g}}-\mathbf{Mod}}
\newcommand\DMod{\mathbf{Dmod}}
\newcommand\IC{\mathbf{IC}}
\newcommand\Gr{\mathbf{Gr}}
\newcommand\semiinf{\frac{\infty}{2}}
\newcommand\KL{\mathbf{KL}}
\newcommand\Whit{\mathbf{Whit}}
\newcommand\neglv{-\kappa}
\newcommand\poslv{\kappa}
\newcommand\dneglv{-\check{\kappa}}
\newcommand\dposlv{\check{\kappa}}

\title{Winter Local Langlands \textsc{\Wh{5}}}

\begin{document}

\begin{abstract}The purpose of this talk (\Wh{5}) is two-fold: first we will set
up the context needed to attack the FLE; second, we will address material needed
for the local-to-global compatibilities to be discussed in \Ja{4}.\end{abstract}

  \maketitle

  % \begin{rmk}
  %   This talk has been cut down due to a change in the planned exposition.
  % \end{rmk}

\section{The formal setting}

  We begin by addressing the formal setting: Recall that we have introduced two
  forms of the FLE (at positive and negative levels
  respectively).\\


  We begin at the negative level\footnotemark; when we set out to prove the FLE
  we want to relate the Kazhdan-Lusztig category for $G$ and Whittaker category
  for $\Gr_{\check{G}}$ to the corresponding categories for the Torus; this
  relation takes the form of a diagram:

  \begin{tikzcd}
    \KL(G)_{\neglv} \arrow[d,"j_!^{KM,Lus}"] \arrow[r,"FLE_G" description]
     & \Whit_{\dneglv}(\Gr_{\check{G}}) \arrow[d, "j_!^{Whit,Lus}"] \\
     \KL(T)_{\neglv} \arrow[r, "\sim", "FLE_T"] & \Whit_{\dneglv}(\Gr_{\check{T}}) \\
  \end{tikzcd}

  \footnotetext{Recall that by convention
    $\kappa$ denotes positive level, hence $-\kappa$ is used for negative
  level.}

  The desired vertical functors should make the diagram commute, be factorizable
  and by applying these functors to the units on both sides we expect to obtain
  the factorization algebras constituting the subject of the last talk.
  Morevover, the factorizable algebras should match up under the bottom
  isomorphism, but it will \textbf{not} be the case that the induced map on
  factorization module categories will be an equivalence.
  % insert some discussion

  The essential thrust of the last talk will be a conjectural bootstrapping of
  an equivalence given what structure we do have apparent here.

  \begin{rmk}
    We saw in \Ja{3} the definition of the functor $j_!^{KM,Lus}$. Note the
    designation of \textit{Lus} in contrast to the existence of a more naive
    functor corresponding to the de-Concini-Kac form of the quantum group, which
    will not produce the correct diagram. %more commentary
  \end{rmk}

  % TODO check to see if Lin's notes defines both Whit and KM sides

  We still need a definition of the functor $j_!^{Whit,Lus}$. First, we'll drop
  the notation \textit{Lus}\footnotemark\footnotetext{reserving this notation
    for the factorization algebra as in $\Omega_q^{Lus}$ arising as the image of
  the unit under the functor $j_!^{KM}$} and define another functor
  $j_{!*}^{Whit}$; this new functor also
  has an incarnation on the Kac-Moody side as $j_{!*}^{KM}$, giving a diagram:

  \begin{tikzcd}
    \KL(G)_{\neglv} \arrow[transform
    canvas={xshift=1.9ex},d,"j_{!*}^{KM}"]\arrow[transform
    canvas={xshift=-1.9ex}, d,"j_!^{KM}"] \arrow[r,"FLE_G" description]
    & \Whit_{\dneglv}(\Gr_{\check{G}}) \arrow[transform
    canvas={xshift=1.9ex},d, "j_{!*}^{Whit}"]\arrow[transform
    canvas={xshift=-1.9ex}, d, "j_!^{Whit}"] \\
     \KL(T)_{\neglv} \arrow[r, "\sim", "FLE_T"] & \Whit_{\dneglv}(\Gr_{\check{T}}) \\
  \end{tikzcd}

  We can similarly express the corresponding diagram at positive level:

  \begin{tikzcd}
    \KL(G)_{\poslv} \arrow[transform
    canvas={xshift=1.9ex},d,"j_{!*}^{KM}"]\arrow[transform
    canvas={xshift=-1.9ex}, d,"j_!^{KM}"] \arrow[r,"\tilde{FLE}_G" description]
    & \Whit_{\dposlv}(\Gr_{\check{G}}) \arrow[transform
    canvas={xshift=1.9ex},d, "j_{!*}^{Whit}"]\arrow[transform
    canvas={xshift=-1.9ex}, d, "j_!^{Whit}"] \\
    \KL(T)_{\poslv} \arrow[r, "\sim", "\tilde{FLE}_T"] & \Whit_{\dposlv}(\Gr_{\check{T}}) \\
  \end{tikzcd}

  As we have seen, for the torus there is no disctinction between FLE for
  positive and negative levels (recall positivity refers to the killing form on
  simple factors.) However, on the Kac-Moody side, these functors will differ by
  a Cartan involution. In contrast, since the Whittaker category is geometric,
  it does not see the distinction between positive and negative level; the main
  features of these functors on the Whittaker side will be the same as at the
  negative level. As we saw in \Ja{3}, these functors on the Kac-Moody
  side strongly depend on the positivity or negativity of the level, while on
  the Whittaker side they do not.\\

  \begin{rmk}
    What was referred to as $j_*$ in \Ja{3} (in relation to $C_*$ semi-infinite
    cohomology) at positive level is here $j_!$.
    % TODO clarify
  \end{rmk}

  In the next section we will recall definitions of a subset of these functors
  (for the rest, and in greater detail, see \Ja{3}).

  \begin{rmk}
    We can express the plan for \Ja{4} in terms of these FLE diagrams for
    negative and positive level: to see this, note that all of these local
    categories are related to the corresponding global categories
    ($\Bun(G),\Bun(\check{G}) \Bun(T),\Bun(\check{T})$, and similarly for
    parabolics and their levi factors etc.\ ), such that there analogously
    exist functors relating these global categories. \Ja{4} will describe the
    interaction of these global categories equipped with global functors and
    their compatabilities and interactions with variants of the Eisenstein
    series and constant term functors. We will see that the structure amounts
    in the global setting to a series of commutative cubes.
  \end{rmk}

\section{Definitions of the functors $j_!$ and $j_{!*}$ on the Kac-Moody Side}
\subsection{$j_{\{!,!*\}}^{KM}$ at the positive level}
Let us begin by recalling from \Ja{3} the
semi-infinite cohomology functor

  $$C_* = C^{\semiinf}(\Lnm^+, -),$$

  in terms of which we can define $j_!^{KM}$ and $j_{!*}^{KM}$, via the following diagram:

% TODO swap to more traditional N^-(K)T(O) notation?
  \begin{tikzcd}
    j_{\{!,!*\}}^{KM} : \affgmod_{\poslv}^{\LG^+} \ar[equal, d] \arrow[r, transform
    canvas={yshift=1.9ex}, "\phi_!"] \arrow[r, transform
    canvas={yshift=-1.9ex}, "\phi_{!*}"]&
    (\affgmod_{\poslv})_{\LNm\LT^+} \arrow[r, "C_*"] &
      (\afftmod_{\poslv})_{\LT^+} \ar[equal, d] \\
    \KL(G)_{\poslv}  & & \KL(T)_{\poslv} \\
  \end{tikzcd}

  Here we have used the fact that invariants and co-invariants (conjecturally)
  coincide via an equivalance, as well as the fact that $C_*$ is well-behaved at
  the positive level. Thus, at the positive level we have reduced the definition
  of $j_!^{KM}$ and $j_{!*}^{KM}$ to the specification of
  $\phi_!$ and $\phi_{!*}$.

\subsection{The functors $\phi_{!}$ and $\phi_{!*}$ by convolution}
The functors $\phi_!$ and $\phi_{!*}$ map $\LG+$
invariants to $\LNm\LT^+$-coinvariants, and hence exhibit a universal nature:
both are given by convolution with an object of

  \begin{tikzcd}
    \DMod(\Gr_G)_{\LNm\LT^+} \arrow[r,"\sim"] & \DMod(\Gr_G)^{\LN\LT^+}.
  \end{tikzcd}

  The appearance of the affine grassmannian as underlying geometry for our
  $\DMod$ category corresponds to the fact that in the domain of the functors
  $\phi_{\{!,!*\}}$ we have passed to
  $\LG^+$-invariants; by the action of convolution by anything in this category,
  we can pass to $\LNm\LT^+$-coinvariants so as to land in the desired codomain.

  \begin{rmk}
    Given that we do not yet have the factorizable equivalence between
    invariants and co-invariants used above, it is worth noting that we will
    only need the functor in one direction, namely, the actual diagram we need
    is
  \begin{tikzcd}
    \DMod(\Gr_G)_{\LNm\LT^+} & \arrow[l] \DMod(\Gr_G)^{\LN\LT^+}.
  \end{tikzcd}
  \end{rmk}

  In order to specify these two functors, we can thus specify two factorizable
  objects we will convolve against. To this end, consider the inclusion

  \begin{tikzcd}
    S_{Ran}^0 \arrow[r,hookrightarrow,"j"] & \overline{S_{Ran}^0} \\
  \end{tikzcd}

  of the zero semi-infinite orbit into its closure (where Ran here again denotes
  the factorizable/global version). We
  have an inclusion of the corresponding $\DMod$ categories:

  \begin{tikzcd}
    \DMod(\Gr_G)_{\LNm\LT^+} & \arrow[l] \DMod(\Gr_G)^{\LN\LT^+}. \\
    & \DMod(\overline{S^0})^{\LN\LT^+} \arrow[u,hookrightarrow]\\
  \end{tikzcd}

  One of our objects will nominally be built from the dualizing object.

  \begin{rmk}

  Recall that in order to even define a restriction functor
  over the Ran space---which is a colimit---we have
  to consider maps

  \begin{tikzcd}
    S_{X^I}^0 \arrow[d] \\
    X^I
  \end{tikzcd}

  for finite sets $I$ over a smooth curve $X$. In general, there is no base
  change between shriek-pushforward and shriek-pullback, but one can prove a
  theorem here that in fact base-change is well-defined; $j_!$ over the entire
  Ran space will in effect realize $j_!$ over any given $X^I$---i.e.\ $j_!$
  behaves well with respect to the given pullback diagrams, hence our desired
  object is factorizable.
  \end{rmk}

  The second object will be the semi-infinite $\IC^{\semiinf}$ sheaf introduced
  in a previous talk, hence our two functors will be built from convolution
  against:

  \begin{tikzcd}
    %S_{Ran}^0 \arrow[r,hookrightarrow,"j"] & \overline{S_{Ran}^0} \\
    j_!(\omega_{S_{Ran}^0}) & \IC^{\semiinf} \\
  \end{tikzcd}

  \begin{rmk}
    The $j_!$ functor (at the positive level) realizes semi-infinite cohomology:
    to see this, note we took the object yielding our $j_!$ functor, considered it
    as part of the $\LN\LT^+$ invariant category, moved it to the $\LNm\LT^+$
    co-invariant category and then convolved against it. This produces a functor
    equivalent to forgetting $\LG^+$ equivariance followed by projection.
  \end{rmk}

  Next we will define the analagous functors on the Whittaker category.


  \begin{rmk}
  We have not said anything on the a-priori conceptual reason for expecting these
  functors as-defined on either side (KM and Whittaker) to match up? One
  possible answer: follow local-to-global compatabilities. Another way might be
  to see it via the original 2-categorical formalism.
  \end{rmk}

\subsection{$j_{\{!,!*\}}^{whit}$ on the Whittaker side}

We want a functor from the Whittaker category on the affine
grassmannian\footnotemark\footnotetext{As remarked earlier, since the Whittaker
category is geometric the twisting parameter will play no role} to the Whittaker
category for the affine grassmannian of the
torus\footnotemark\footnotetext{Since the torus has no unipotent part, the Whit
functor is vacuous}.

  \begin{tikzcd}
    \Whit_{\poslv}^!(\Gr_{\check{G}}) \arrow[d,transform
    canvas={xshift=1.9ex},"j_!^{whit}"] \arrow[d,transform
    canvas={xshift=-1.9ex},"j_{!*}^{whit}"] \\  %\arrow[d,"j_!^{whit}, j_{!*}^{whit}"] \\
    \DMod(\Gr_{\check{T}})
  \end{tikzcd}

  \begin{rmk}
  Once again, we must reproduce the analogous structure from the Kac-Moody side
  factorazably. Here we will only indicate the structure at a point, but since
  the objects introduced posess a factorization structure, it should be clear
  how to extend the structure over the Ran space.
  \end{rmk}

  The affine grassmannian for the torus $\check{T}$ is a discrete set indexed by
  the coweights of $\check{T}$; hence we will define our functors $j_{\{!,!*\}}$
  separately for each coweight.

  \subsection{$j_!^{whit,\lambda}$ and $j_{!*}^{whit,\lambda}$}
  Recall our underlying geometry of sheaves over the affine grassmannian with an
  equivariance condition. Once again we start with the zero semi-infinite orbit
  $S^0$, but in addition we will also consider the various translates by
  coweights $\lambda$, denoted $S^{0,\lambda}$. In lieu of $\LN$, we need to
  consider equivariance with respect to $\LNm$, which we'll denote by
  $S^{-,\lambda}$.  The analogous inclusion of orbits into their closures, along
  with the two sheaves we will consider for convolution, becomes:

  \begin{tikzcd}
    S^{-,\lambda} \arrow[r,hookrightarrow,"j"] & \overline{S^{-,\lambda}} \\
    j_!(\omega_{S^{-,\lambda}}) & \IC^{\semiinf+\lambda} \\
  \end{tikzcd}

  We stress that we have translated the analogous objects from the zero-semi-infinite
  orbit; in particular, here $\IC^{\semiinf+\lambda}$ is the fiber of the sheaf introduced in
  \Ja-5, restricted to a copy of the affine grassmannian over one point
  of the curve.

  \begin{rmk}
    We warn again that $\IC^{\semiinf+\lambda}$ is \textbf{not} the $\IC$-extension in the t-structure of a
    single copy of the grassmannian. One needs to perform the extension globally
    over the Ran space (i.e.\ allowing the points to move), followed by
    restriction to the fiber over a given point of the curve and a subsequent
    shift by $\lambda$.
  \end{rmk}

  The functors $j^{whit}_{\{!,!*\}}$ are then given by mapping any given
  Whittaker sheaf $\mathcal{F}$ to the de Rahm cohomology along the affine
  grassmannian of the shriek-tensor-product of $\mathcal{F}$ with either of the
  two objects introduced above.

  \begin{tikzcd}
    \mathcal{F} \mapsto & \Gamma_{dR}(\Gr_G, \mathcal{F} \otimes^! -)
  \end{tikzcd}

  In particular, starting with an $\LN$ equivariant sheaf (hence a geometric
  object with truly infinite dimensional support), we intersect it with
  something equivariant with respect to $\LNm$. On compact objects this has the
  effect of producing a geometric object supported on the closure of the
  intersection of particular $\LN$ and $\LNm$ orbits. This intersection is
  finite dimensional, hence the above de Rahm cohomology is taken over a finite
  dimensional variety.

  These are the Jacquet functors.

\end{document}
