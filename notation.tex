%% Index into Symbols and Notation
%% We collect all notation utilized throughouth this lecture series
%% we the hope their collation facillitates future functional/aesthetic changes

%%%%%%%%%%%%%%%%%%%%%%%%%%%%%%%%%%%%
%% TODO
%% - clarify class formation group k
%%    * little/big, more readable
%%    * visually distinguish notation for C_L / element
%% - Generate a glossary
%% - mark up structure for nomenclature / glossary inline
%%

\nc\Cat[1]{\mathbf{#1}} % typeset name of category
% TODO fit these in
\nc\G{G}           % reductive group
% algebraic fields, formal power series and their fraction fields
\nc\ringint{\mathcal{O}}           % abstract ring of integers
\nc\grk{k}                         %abstract ground field
\nc\N{\mathbb{N}}
\nc\Z{\mathbb{Z}}
\nc\Q{\mathbb{Q}}
\nc\R{\mathbb{R}}
\nc\C{\mathbb{C}}
\nc{\FnS}{\grk\lp\Sigma\rp}
\nc\laurentpoly{\lb x,x^{-1}\rb}
\nc\laurentser{\llp~z\rrp}
\nc\fpowerser{\llb~z\rrb}
\nc\klocint{\grk\fpowerser}
\nc\Klocfrac{K\laurentser}
\nc\tensor{\otimes} % algebraic tensor product

%\nc\Spec[1]{\mathrm{Spec} (#1)}
%\nc\Specf[1]{\mathrm{Specf} (#1)}

% TODO figure out optimal manner of handling infix macros
% something like \infixnewcommand{\T}{{bunewcommandhofstuff} {#1} {bunewcommandhofstuff} {#2} {bunewcommandhofstuff}}

% TODO read more about macros
% https://en.wikibooks.org/wiki/LaTeX/Macros

% relations, operations, unitary and binary symbols etc.
\nc\isom{\simeq} % TODO find optimal

\nc\ov[1]{/ {#1}}%'over' as in Scheme over k or galois extension over k

\nc\restr[2]{{% we make the whole thing an ordinary symbol
  \left.\kern-\nulldelimiterspace% automatically resize the bar with \right
  #1% the funewcommandtion
  \vphantom{\big|} % pretend it's a little taller at normal size
  \right|_{#2} % this is the delimiter
  }}

\nc\shf[1]{\mathcal{#1}}
\nc\idshf[1]{\mathcal{#1}}

\nc\projline{\mathbb{P}^1}

\nc\K{\mathbb{K}}

\nc\polyring[2]{R}

\nc\classgrp[1]{C_{#1}}
\nc\F{\mathbb{F}}
\nc\idealgenby[1]{\langle#1\rangle}
\nc\D{\mathbf{d}}
\nc\pt{\mathbf{pt}}
\nc\Gr{\textrm{Gr}} % affine grassmannian.

  \nc\Ga{\mathbb{G}_a}

  \nc\Gm{\mathbb{G}_m}

\nc\lpgrass{\mathcal{G}}
\nc\glpgrass{\mathcal{G}^P (I,Q)}

%, commonly denoted $\Gr_a$ or $G_{ K / {\ringint}$.}
% TODO = \mathbb{G}_{\mathcal{G}} = \mathcal{G}_{\mathcal{K}} \textrm{
% mod } \mathcal{G}_{\mathcal{O}}
% TODO structure nomenclature pre-amble by
% TODO 1. generality / importanewcommande
% TODO 2. chapter/lecture

% TODO group + order intro

%%%%%%%%%%%%%%%%%%%%%
%% classformations %%
%%
\nc\Flds{\Cat{Fields}}
\nc\cartdual{\mc{D}}

\nc\Mat[1]{\mathbf{Mat}_{#1\times#1}}
\nc\RH{\mathbb{\R}\textrm{H}}
\nc\tateshiftedby[1]{\left[#1\right]}
\nc\TateCat{\Cat{J_B}}
\nc\TateCatG{\Cat{J_G}}
\nc\B[1]{\mathbb{B} (#1)}
\nc\ptmod[1]{\mathbf{pt}\ov{#1}}
\nc\Ind{\mathrm{Ind}}
\nc\Hom{\mathrm{Hom}}
\nc\Map{\mathrm{Map}}
\nc\coH{\mathbf{H}}
\nc\hcoH[3]{\widehat{\coH^{#1}#2,#3}}
\nc\Ab{\Cat{Ab}} % TODO fix
\nc\Ker{\mathbf{Ker}}
\nc\Coker{\mathbf{Coker}}
\nc\dg{\textrm{DG}}
\rc\Vec{\Cat{Vec}}
\nc\VecF{\Cat{Vec}_{\grk}}
\nc\Coh{\Cat{Coh}}

\nc\Coind{\mathbf{Coind}}
\rc\Ind{\mathbf{Ind}}
\nc\Res{\mathbf{Res}}

%%%%%%%%%%%%%%%%%%%%%
%% classformations %%
%%
\nc\Pic{\mathbf{Pic}}
\nc\AJ{\mathbf{AJ}}
\nc\divisor[1]{\mathbf{#1}}
\nc\divisoridshf[1]{\idshf{I_{\divisor{#1}}}}
\nc\IndProSch{\Cat{IndProSch}}
\nc\FinIndProSch{\Cat{FinIndProSch}}
\nc\Pro{\mathbf{Pro}}
\nc\IndPro{\Ind-\Pro}
\nc\FinIndSch{\Cat{FinIndProSch}}
\nc\A{\Cat{CommIndSch_{\grk}}}

\nc\Fun[2]{#1 (#2)}
\nc\Frac[1]{\mathbf{Frac} (#1)}
\nc\curriedleftarg{\ldots}
\nc\Zp{\Z_p}

\nc\Qp{\mathbb{Q}_p}

% TODO group + order lec 3
% TODO review cardinality issues / small categories etc.
\nc\Set{\Cat{Set}} % Category of Sets

\nc\Hilb{\mathcal{H}\mathbb{ilb}}

\nc\disj{\sqcup} % TODO distinguish bigsqcup and sqcup
\nc\bij{\leftrightarrow}
\nc\sur{\twoheadedrightarrow}
\nc\inj{\rightarrowtail}
\nc\injects{\inj}
% \nc\xinj{\xrightarrowtail} % see sty-glg/commands package
% \nc\xhkinj{\xrightarrowtail} % see sty-glg/commands package
% \begin{document}
%   \[ f : G \xrightarrowtail[{\star}]{\text{\textbf{Grp}}} H \]
% \end{document}

% \xhookrightarrow from mathtools
% \[
% A\xhookrightarrow{} B\qquad A\xhookrightarrow{f\cirenewcommand g} B\qquad
% A \xhookrightarrow[(f\cirenewcommand g)\cirenewcommand h]{} B
% \]
% \xrightarrow from amsmath
% \[
% A\rightarrow{} B\qquad A\xrightarrow{f\cirenewcommand g} B\qquad
% A \xrightarrow[(f\cirenewcommand g)\cirenewcommand h]{} B
% \]

% TODO group, order + split off cohomology.tex

\nc\fst{\ensuremath{2^{\text{\tiny{st}}}}}
\nc\snd{\ensuremath{3^{\text{\tiny{nd}}}}}
\nc\thrd{\ensuremath{3^{\text{\tiny{rd}}}}}
\nc\nth{\ensuremath{n^{\text{\tiny{th}}}}}
% TODO find elegant resolution; possibilities:
% * \(n\)th
% * \usepackage{nth} NB: \nth{3} is different from $\nth{3}$
% * $n$-th
% * $n^{\text{\tiny th}}$
\nc\shfcoH[3]{\mathbf{H}^{#3} (#1,#2)}
\nc\hshfcoH[3]{\widehat{\shfcoH{#1,#2,#3}}}
\nc\Obj{\mathbf{Obj}}
% TODO nomencl Obj
\nc\grpring[2]{#1 \left[ #2 \right]}

\nc\Ext{\mathbf{Ext}}
\nc\Exts[3]{\Ext_{#1}^*\left(#2,#3 \right)}
% TODO nomencl Exts
\nc\dualmod[1]{#1^{*}}
% TODO investigate prefix macros e.g. {A}\Mod
\nc\LModCat[1]{#1-\Cat{LMod}}
\nc\RModCat[1]{#1-\Cat{RMod}}
\nc\indlim{\mathrm{lim}}

% TODO group + order indobjs.tex

\nc\Schemes{\Cat{Schemes}}

\nc\NoethSch{\Cat{NoethSch}}
\nc\NoethIntSch{\Cat{NoethIntSch}}

\nc\Sch[1]{\Cat{Sch}_{#1}}
\nc\IndSch[1]{\Cat{IndSch}_{#1}}

\nc\ModCat[1]{#1-\Cat{Mod}}

%\rc\dim[1]{\ensuremath{\mathbf{dim} (#1)}}

%%%%%%%%%%%%%%%%%%%%%%%%%%
%%
%%
% TODO learn how to typeset arrows in categories
% functors / presheaves / sheaves / group schemes etc.

\nc\grkAlg{\Cat{\grk-Algebras}}
\nc\ZAlg{\Cat{\Z-Algebras}}
%\nc\SL[1]{\mathbf{SL}_{#1}}

%{\ifstrequal{#1}{}{%if no scale provided
%  \SpecialLinearAbr}{


%% Index into Symbols and Notation
%% We collect all notation utilized throughouth this lecture series
%% we the hope their collation facillitates future functional/aesthetic changes

%%%%%%%%%%%%%%%%%%%%%%%%%%%%%%%%%%%%
%% TODO
%% - clarify class formation group k
%%    * little/big, more readable
%%    * visually distinguish notation for C_L / element
%% - Generate a glossary
%% - mark up structure for nomenclature / glossary inline
%%

% TODO fit these in
\nc\cc
{\ensuremath{\mathbf{CC}}} % TODO fix this
\nc\Bun{\ensuremath{\mathrm{Bun}}}

\nc\Spec[1]{\mathrm{Spec} (#1)}
\nc\Specf[1]{\mathrm{Specf} (#1)}

\nc\mK{\mathbf{K}} % TODO abstract out to notation.tex
\nc\Bil{\mathbf{Bil}} % TODO abstract to notation.tex
\nc\IndNSt{\Cat{Indn-Stack}}
% TODO figure out optimal manner of handling infix macros
% something like \infixnewcommand{\T}{{bunewcommandhofstuff} {#1} {bunewcommandhofstuff} {#2} {bunewcommandhofstuff}}

% TODO read more about macros
% https://en.wikibooks.org/wiki/LaTeX/Macros

% relations, operations, unitary and binary symbols etc.

%%%%%%%%%%%%%%%%%%%%%
%% classformations %%
%%
\rc\Pic{\mathbf{Pic}}
\rc\AJ{\mathbf{AJ}}
\rc\divisor[1]{\mathbf{#1}}
\rc\divisoridshf[1]{\idshf{I_{\divisor{#1}}}}
\rc\IndProSch{\Cat{IndProSch}}
\rc\FinIndProSch{\Cat{FinIndProSch}}
\rc\Pro{\mathbf{Pro}}
\rc\IndPro{\Ind-\Pro}
\rc\FinIndSch{\Cat{FinIndProSch}}
\rc\A{\Cat{CommIndSch_{\grk}}}

\rc\Fun[2]{#1( #2 )}
\rc\Frac[1]{\mathbf{Frac}( #1 )}
\rc\curriedleftarg{\ldots}
\rc\Zp{\Z_p}

%also called the formal completion at the point $p$ in the curve $\Spec{\Z}$}
\rc\Qp{\mathbb{Q}_p}

\rc\Hilb{\mathcal{H}\mathbb{ilb}}

% \rc\bij{\leftrightarrow}
% \rc\sur{\twoheadedrightarrow}
\rc\inj{\rightarrowtail}
\rc\injects{\inj}
% \rc\xinj{\xrightarrowtail} % see sty-glg/commands package
% \rc\xhkinj{\xrightarrowtail} % see sty-glg/commands package
% \begin{document}
%   \[ f : G \xrightarrowtail[{\star}]{\text{\textbf{Grp}}} H \]
% \end{document}

% \xhookrightarrow from mathtools
% \[
% A\xhookrightarrow{} B\qquad A\xhookrightarrow{f\cirenewcommand g} B\qquad
% A \xhookrightarrow[(f\cirenewcommand g)\cirenewcommand h]{} B
% \]
% \xrightarrow from amsmath
% \[
% A\rightarrow{} B\qquad A\xrightarrow{f\cirenewcommand g} B\qquad
% A \xrightarrow[(f\cirenewcommand g)\cirenewcommand h]{} B
% \]

% TODO group, order + split off cohomology.tex

\rc\fst{\ensuremath{2^{\text{\tiny{st}}}}}
\rc\snd{\ensuremath{3^{\text{\tiny{nd}}}}}
\rc\thrd{\ensuremath{3^{\text{\tiny{rd}}}}}
\rc\nth{\ensuremath{n^{\text{\tiny{th}}}}}
% TODO find elegant resolution; possibilities:
% * \(n\)th
% * \usepackage{nth} NB: \nth{3} is different from $\nth{3}$
% * $n$-th
% * $n^{\text{\tiny th}}$
\rc \shfcoH[3]{\mathbf{H}^{#3}(#1,#2)}
\rc\hshfcoH[3]{\widehat{\shfcoH{#1,#2,#3}}}
\rc\Obj{\mathbf{Obj}}
% TODO nomencl Obj
\rc\grpring[2]{#1 \left[ #2 \right]}

\rc\Ext{\mathbf{Ext}}
\rc\Exts[3]{\Ext_{#1}^*\left(#2,#3 \right)}
% TODO nomencl Exts
\rc\dualmod[1]{#1^{*}}
% TODO investigate prefix macros e.g. {A}\Mod
\rc\LModCat[1]{#1-\Cat{LMod}}
\rc\RModCat[1]{#1-\Cat{RMod}}
\rc\indlim{\mathrm{lim}}

% TODO group + order indobjs.tex

\rc\Schemes{\Cat{Sch}}

\rc\NoethSch{\Cat{NoethSch}}
\rc\NoethIntSch{\Cat{NoethIntSch}}

\rc\Sch[1]{\Cat{Sch}_{#1}}

\rc\IndSch[1]{\Cat{IndSch}_{#1}}

\rc\ModCat[1]{#1-\Cat{Mod}}
%commutative ring $A$.}

\rc\dim[1]{\ensuremath{\mathbf{dim}( #1 )}}

% TODO learn how to typeset arrows in categories
% functors / presheaves / sheaves / group schemes etc.

\rc\grkAlg{\Cat{\grk-Algebras}}
\rc\ZAlg{\Cat{\Z-Algebras}}
\nc\SL{\mathbf{SL}}
\nc\GL{\mathbf{GL}}
\nc\DMod{\Cat{DMod}}
\nc\IndSt{\Cat{IndStacks}}
\rc\IndNSt{\Cat{Ind}-\textsc{N}-\Cat{Stacks}}

\nc\Curve{\Sigma}
\nc\IxCHilb{\Hilb_{\Curve \times I}}
\nc\HilbCxI{\Hilb_{\Curve \times I}}
\nc\locLb{\mc{L}_{I,Q}}
\nc\Locvbpos{\{V_p\}_{p \in P}}
%\nc\IxCHilb{\Hilb_{\Curve \times I}}
%\nc\locLb{\mc{L}_{I,Q}}
